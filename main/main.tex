% !Mode:: "TeX:UTF-8"
\documentclass{article}
\input{en_preamble.tex}
\input{xecjk_preamble.tex}
\setCJKmainfont{STKaiti} % 如果请替换为本地系统有的字体
%中文断行
\XeTeXlinebreaklocale "zh"
\XeTeXlinebreakskip = 0pt plus 1pt minus 0.1pt
\begin{document}
\title{共轭梯度法}
\author{}
\date{\today}
\maketitle
%\tableofcontents
%\newpage
\subsection{共轭梯度法解方程的算法流程}

例1: 设方程组
\begin{equation}\label{eq:AB}
\begin{bmatrix}
6 & 3\\
3 & 2\\
\end{bmatrix}
\begin{bmatrix}
x_1\\
x_2\\
\end{bmatrix}
=\begin{bmatrix}
0\\
-1\\
\end{bmatrix}
\end{equation}
取 $x^{(0)}=[0,0]^T$, 试用共轭梯度法求解方程组.

解:共轭梯度法解方程组的计算程序如下
取一个初始向量 $x_0$, 计算 $x_0=Ax_0-b$, 取 $g0=r0$
\begin{equation}\label{eq:AC}
\left\{
\begin{array}{l}
\alpha_k = \frac{(r_k,r_k)}{(Ag_K,g_k)}\\
X_{k+1} = X_k+\alpha_k g_k,r_{k+1}=r_k+\alpha_kAg_k,k=0,1,2,\cdots\\
\lambda_k=\frac{(r_{k+1},r_{k+1})}{(r_K,r_k)}\\
g_{k+1}=r_{k+1}+\lambda_kg_k\\
\end{array}\right.
\end{equation}

具体计算过程

\begin{equation}
\begin{bmatrix}
6 & 3\\
3 & 2\\
\end{bmatrix}
\begin{bmatrix}
x_1\\
x_2\\
\end{bmatrix}
=\begin{bmatrix}
0\\
-1\\
\end{bmatrix}
\end{equation}

其中 

$$
A=\begin{bmatrix}
6 & 3\\
3 & 2\\
\end{bmatrix},
\quad
b=\begin{bmatrix}
x_1\\
x_2\\
\end{bmatrix}
$$

取 $x_0=x^{(0)}=[0,0]^T$, 则代入\eqref{eq:AB}式得

$$
\alpha_0=-\frac{(r_0,r_0)}{(Ag_0,g_0)}=\frac{
\begin{bmatrix}
0 & 1\\
\end{bmatrix}
\begin{bmatrix}
0 \\
1\\
\end{bmatrix}
}{\begin{bmatrix}
0 & 1\\
\end{bmatrix}
\begin{bmatrix}
6 & 3\\
3 & 2\\
\end{bmatrix}
\begin{bmatrix}
0 \\
1\\
\end{bmatrix}
}=-\frac{1}{2}
$$

$$
X_1 = X_0+\alpha_0 g_0=[0,\frac{1}{2}]^T
$$

$$
r_1=r_0+\alpha_0Ag_0=[-\frac{3}{2},0]^T
$$

$$
\lambda_k=\frac{(r_1,r_1)}{(r_0,r_0)}=\frac{9}{4}
$$

$$
g_1=r_1+\lambda_0g_0=[-\frac{3}{2},\frac{9}{4}]^T
$$

进一步计算可得

$$
\alpha_1=-\frac{(r_1,r_1)}{(Ag_1,g_1)}=-\frac{2}{3}
$$

$$
X_2 = X_1+\alpha_1 g_1=[1,-2]^T
$$

$$
r_2=r_1+\alpha_1Ag_1=[0,0]^T
$$

$$
\lambda_1=\frac{(r_2,r_2)}{(r_1,r_1)}=0
$$

$$
g_2=r_2+\lambda_1g_1=[0,0]^T
$$

则可知方程组的解 $x=X_2 = [1,-2]^T$.
\newpage
\nocite{*}
\bibliography{ref}
\end{document}

